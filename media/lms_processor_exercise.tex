
%
%
% Module: lms_processor_exercise
%
% Author: Michael Kramer
%
%

As with your \matlab simulation, use the designed
elliptical IIR filter as the unknown system in your
DSP implementation.

Although the coefficient update equation is relatively
straightforward, we strongly suggest that you look into
the \verb+lms+ instruction available on the TI processor,
as it is designed for just such an application and yields
a very efficient implementation of the coefficient update
equation.

To generate noise on the DSP you can use your PN generator
code from Lab 5, but be sure to shift the PN register contents
up to keep the sign bit random.  (If the sign bit is always zero
then the noise will not be zero-mean and will affect error
convergence.)  The function generators in the lab can also 
create a noise signal.  It is recommended that you output the desired
signal, $y(n)$, the output of the adaptive filter, $\hat{y}(n)$,
and the error to the D/A for display on the oscilloscope.

When using the step-size suggested in the \matlab simulation
section you should notice that the error converges {\em very} quickly.
You will want to try an extremely small $\mu$ so that you can
actually watch the error signal decrease towards zero in amplitude.

