\documentclass[11pt]{handout}
\usepackage{moreverb}
\usepackage{epsf}
\usepackage{epic}
\usepackage{eepic}

\renewcommand{\coursetitle}{ECE 320}
\renewcommand{\handouttitle}{Adaptive Filtering}
\renewcommand{\handoutauthor}{Michael Kramer}
\renewcommand{\semestertitle}{Spring 1999}

\newcommand{\bea}{\begin{eqnarray}}
\newcommand{\eea}{\end{eqnarray}}

\setlength{\parindent}{5mm}
\begin{document}

\setlength{\baselineskip}{0.5cm}
\setlength{\parskip}{0.5cm}

\makeboxtitle
\vspace{0.3cm}

\section{Introduction:}

This exercise is intended as an introduction to adaptive
filtering.


%
%
% Module: lms_theory_tutorial
%
% Author: Michael Kramer
%
% Reference: S. Haykin, Adaptive Filter Theory.  Prentice Hall, 3rd ed.,
% 1996.
%
%

Figure \ref{fig: sys_id} shows a block diagram
for the use of adaptive filtering for system
identification.  The objective of the system
is to adapt an FIR filter (AFIR) to match as closely
as possible the response of an unknown filter, $H?$.
Both the unknown system and thh adapting filter
are fed with the same input, and each have
respective outputs, $y(n)$ (also referred to as the
desired signal) and $\hat{y}(n)$.

\begin{figure}[htb]\centerline  {
\epsffile{sys_id.eps}  }
\caption{System identification block diagram.}
\label{fig: sys_id}
\end{figure}

\paragraph{Gradient Decent Adaptation:}
The FIR filter is adapted using the least mean-square algorithm.
First the error signal is computed, $e(n) = y(n) - \hat{y}(n)$,
which provides a measure of how far our FIR filter is from
the unknown system output.
The coefficient update relation is a function of this
error signal squared and is given by
\bea
h_{new}(i) & = & h_{old}(i) + \frac{\mu}{2} \left( -\frac{\partial}{\partial
h(i)}
| e | ^2 \right)
\eea

The term inside the parenthesis represents the derivative, or gradient,
of the squared-error with respect to the $i'th$ coefficient, and
the $\mu$ term represents a step-size, or how much gradient
information is used to update each coefficient.
After repeatedly adjusting each coefficient in the direction
opposite to the gradient of the error, the adaptive filter
should converge; that is, the difference between the
unknown and adaptive systems should get smaller and smaller.

To express the gradient decent coefficient update equation
in a more usable manner, we can rewrite the derivative of the
squared-error term as
\bea
\frac{\partial}{\partial h(i)} |e|^2 & = &
2 \left( \frac{\partial}{\partial h(i)} e \right) e \nonumber \\
& = & 2 \left( \frac{\partial}{\partial h(i)} \left[
y - \hat{y} \right] \right) e \nonumber \\
& = & 2  \left( \frac{\partial}{\partial h(i)} \left[
y - \sum_{i=0}^{N-1} h(i) x(n-i) \right] \right) e \nonumber \\
& = & 2 \left( - x(n-i) \right) e
\eea
which in turn gives us the final LMS coefficient update,
\bea
h_{new}(i) & = & h_{old}(i) + \mu \: e \: x(n-i)
\label{eq: lms_update}
\eea
The $\mu$ term, or step-size, directly
affects how quickly the adaptive filter will converge toward
the unknown system.  If $\mu$ is very small, then the coefficients
are not altered by a significant amount at each update.  With
a large step-size, more gradient information is included in
each update; however, when the step-size is too large the
coefficients may be changed too much and the filter will
not converge.




\section{Matlab Simulation:}


%
%
% Module: lms_matlab_exercise
%
% Author: Michael Kramer
%
%

Simulate the system identification block diagram
shown in Figure \ref{fig: sys_id} on a sample
by sample basis; that is, because the FIR filter
changes at each sample, you must use a ``\verb+do+''
loop in \matlab rather than the \verb+conv+ or \verb+filter+
functions, which use the same filter coefficients for the
entire input sequence.  For the unknown system, use the
fourth order low-pass elliptical IIR filter designed for
Lab 2.

Use Gaussian random noise your input, which can
be generated in \matlab using the command ``\verb+randn+''.
Simulate the system with an initial adaptive FIR
of zeros, starting with an adaptive filter of
length 32, and a step-size of $0.02$.
From your simulation you should be able
to plot the error (or squared-error) as it
evolves over time, as well as the final
set of adapted coefficients.  (How do they compare
to the unknown system coefficients?)

With your simulation working, you will then want
to experiment with different step-sizes and
adaptive filter lengths.



\section{Implementation:}


%
%
% Module: lms_processor_exercise
%
% Author: Michael Kramer
%
%

As with your \matlab simulation, use the designed
elliptical IIR filter as the unknown system in your
DSP implementation.

Although the coefficient update equation is relatively
straightforward, we strongly suggest that you look into
the \verb+lms+ instruction available on the TI processor,
as it is designed for just such an application and yields
a very efficient implementation of the coefficient update
equation.

To generate noise on the DSP you can use your PN generator
code from Lab 5, but be sure to shift the PN register contents
up to keep the sign bit random.  (If the sign bit is always zero
then the noise will not be zero-mean and will affect error
convergence.)  The function generators in the lab can also 
create a noise signal.  It is recommended that you output the desired
signal, $y(n)$, the output of the adaptive filter, $\hat{y}(n)$,
and the error to the D/A for display on the oscilloscope.

When using the step-size suggested in the \matlab simulation
section you should notice that the error converges {\em very} quickly.
You will want to try an extremely small $\mu$ so that you can
actually watch the error signal decrease towards zero in amplitude.



\section{Extensions:}

Because your final project will require some modifications to 
the discussed system identification implementation, you
will want to refer to the listed reference, \cite{Haykin1}, 
and consider some of the following questions regarding 
such modificiations:

\begin{itemize}
\item{How would the system in Figure \ref{fig: sys_id} 
change for different applications? (noise cancellation,
equalization, etc.)}
\item{What happens to the error when the step-size is too 
large or too small?}
\item{How well does the error converge for different length 
adaptive FIR filters?}
\item{What other types of coefficient update relations are 
possible other than the described LMS algorithm?}
\end{itemize}

\bibliographystyle{ieeetr}
\bibliography{../../ece320}

%\small
%\ifx\undefined\allcaps\def\allcaps#1{#1}\fi
%\begin{thebibliography}{2}
%
%\bibitem{Haykin}
%S. Haykin, {\em Adaptive Filter Theory}.  Prentice Hall,
%3rd ed., 1996.
%
%\end{thebibliography}

\end{document}
\bye 
